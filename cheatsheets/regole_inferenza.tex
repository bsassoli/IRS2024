\documentclass[12pt]{article}
\usepackage{fix-cm}
\usepackage{enumitem}

% Define document-specific variables before loading template
\newcommand{\DocumentTitle}{Principali Regole di Inferenza}  % This will change for each document

% Load the template
% header-template.tex
\usepackage{graphicx}
\usepackage{lastpage}
\usepackage{amssymb}
\usepackage{amsmath}
\usepackage{fancyhdr}
\usepackage{geometry}
\usepackage[colorlinks=true, linkcolor=blue, urlcolor=blue]{hyperref}
\usepackage{microtype}
% \usepackage{tgpagella}
%\usepackage[T1]{fontenc}
%\usepackage{newtxmath}

% Alternative font closer to University of Milan's official font
\usepackage[cmintegrals,cmbraces]{newtxmath}
\usepackage{ebgaramond-maths}
\usepackage[T1]{fontenc}

% Define default values for variables
\providecommand{\CourseTitle}{Introduzione al ragionamento scientifico}
\providecommand{\AcademicYear}{2024\mbox{-}25}
\providecommand{\Professor}{prof. Bernardino Sassoli de' Bianchi}
\providecommand{\DocumentDate}{Ottobre 2024}
\providecommand{\DocumentTitle}{Regole di inferenza}
\providecommand{\DocumentType}{Cheatsheet}

\geometry{a4paper, margin=1in}
\setlength{\headheight}{46.27916pt}
\addtolength{\topmargin}{-34.27916pt}



% Remove header rule
\renewcommand{\headrulewidth}{0pt}
\renewcommand{\footrulewidth}{0pt}

\fancypagestyle{firstpage}{%
    \fancyhf{} % clear all headers and footers
    \fancyhead[L]{%
        \begin{minipage}[c]{0.5\textwidth}
            \includegraphics[width=3.5cm]{marchio-04.jpg}
        \end{minipage}%
    }
    \fancyhead[R]{%
        \begin{minipage}[c]{0.6\textwidth}
            \raggedleft
            {\footnotesize
            \CourseTitle\\
            \Professor\\
            A.A. \AcademicYear\\
            }
        \end{minipage}%
    }
}

% Define style for other pages
\fancypagestyle{followingpage}{%
    \fancyhf{} % clear all headers and footers
    \fancyhead[R]{\small\DocumentTitle}
    \fancyfoot[R]{\thepage}
}

% Set up the page styles to apply automatically
\AtBeginDocument{%
    \pagestyle{followingpage}% Set default style for all pages
    \thispagestyle{firstpage}% Override first page only
}

% Command for the standard title block
\newcommand{\MakeCustomTitle}[1]{%
    \begin{center}
    \vspace*{2cm}
    {\Large\textbf{#1}\\[0.5cm]
    \DocumentTitle}
    \end{center}
    \vspace{1cm}
}

\begin{document}

% Use the custom title command
\MakeCustomTitle{Cheatsheet}


% Document content goes here
\section*{Regole di inferenza}

    \begin{description}[
        align=left,          % Aligns the descriptions to the left
        leftmargin=0pt,      % Removes the left margin
        itemindent=0pt,      % Removes item indentation
        style=nextline       % Sets the style to wrap the item text
    ]

        \item[Eliminazione della congiunzione ($\wedge$-elim)]
        Da una congiunzione si possono dedurre i due congiunti.
        \[
        \dfrac{
        \begin{array}{l}
        P \wedge Q
        \end{array}
        }{P} \qquad
        \dfrac{
        \begin{array}{l}
        P \wedge Q
        \end{array}
        }{Q}
        \]

        \item[Introduzione della congiunzione ($\wedge$-int)]
        Da due proposizioni si può dedurre la loro congiunzione.
        \[
        \dfrac{
        \begin{array}{l}
        P\\
        Q
        \end{array}
        }{P \wedge Q}
        \]


        \item[Modus ponens (MP)]
        Da un condizionale e l'antecedente si può dedurre il conseguente.
        \[
        \dfrac{
        \begin{array}{l}
        P\\
        P \to Q
        \end{array}
        }{Q}
        \]

        \item[Modus tollens (MT)]
        Da un condizionale e la negazione del conseguente si può dedurre l'antecedente.
        \[
        \dfrac{
        \begin{array}{l}
        \neg P\\
        P \to Q
        \end{array}
        }{\neg P}
        \]


        \item[Eliminazione della doppia negazione (DN-elim)]
        Dalla doppia negazione di una proposizione posso dedurre la proposizione stessa.
        \[
        \dfrac{
        \begin{array}{l}
        \neg \neg P\\
        \end{array}
        }{P}
        \]

        \item[Sillogismo disgiuntivo (SD)]
        Se ho una disgiunzione e la negazione di uno dei due disgiunti, posso dedurre l'altro disgiunto.
        \[
        \dfrac{
        \begin{array}{l}
        P\lor Q\\
        \neg P
        \end{array}
        }{\neg Q}
        \]

        \item[Contrapposizione del condizionale (CC)]
        Posso sempre trasformare un condizionale in un altro condizionale equivalente che ha antecedente e conseguente invertiti e negati.
        \[
        \dfrac{
        \begin{array}{l}
        P\to Q\\
        \end{array}
        }{\neg Q \ to \neg P}
        \]

        \item[Ragionamento per assurdo (ABS)]
        Per dimostrare $Q$, ipotizzo che $\neg Q$. Se da premesse $P_1, P_2, \ldots, P_n, \neg Q$ si può dedurre una contraddizione, allora si può dedurre $Q$.
        \[
        \dfrac{
        \begin{array}{l}
        P_1\\
        P_2\\
        \vdots\\
        P_n\\
        \neg Q\\
        \bot\footnotemark
        \end{array}
        }{Q}
        \]

        \item[Eliminazione del quantificatore universale ($\forall$-elim)]
        Se ho una proposizione che vale per ogni $x$, posso dedurre che vale per un $x$ arbitrario.
        \[
        \dfrac{
        \forall x P(x)
            }
            {P(a)}
        \]
        \item[Introduzione del quantificatore esistenziale ($\exists$-elim)]
        Se so che un predicato vale per un $a$ arbitrario, posso dedurre che esiste un $x$ tale che il predicato vale per $x$.
        \[
        \dfrac{
        P(a)
            }
            {\exists x P(x)}
        \]
        \item[Eliminazione del quantificatore esistenziale ($\exists$-elim)]
        Se so che esiste un $x$ tale che il predicato vale per $x$, posso dedurre che il predicato vale per un $a$ arbitrario.
        \[
        \dfrac{
        \exists x P(x)
            }
            {P(a)\footnotemark}
        \]

    \end{description}
    \footnotetext{Simbolo per indicare una contraddizione.}
    \footnotetext{A patto che $a$ non occorra in $P(x)$, né in nessuna delle premesse $P_1, P_2, \ldots, P_n$, e che non venga di nuovo utilizzata in una nuova applicazione di $\exists$-elim.}
\end{document}